\documentclass[a4paper,10pt]{article}

\usepackage{pseudocode}
\usepackage{ngerman}
\usepackage{amsmath}

\DeclareMathOperator{\oprdiv}{div}
\DeclareMathOperator{\oprshl}{shl}
\DeclareMathOperator{\oprshr}{shr}
\title{Structorizer LaTeX pseudocode Export of DateTests563.arrz}
% Structorizer version 3.32-26
\author{Kay G"urtzig}
\date{06.02.2025}

\begin{document}

\begin{pseudocode}{isLeapYear}{year }
\label{isLeapYear}
\COMMENT{ Detects whether the given year is a leap year in the Gregorian calendar }\\
\COMMENT{ (extrapolated backwards beyonds its inauguration) }\\
\PROCEDURE{isLeapYear}{year}
  \COMMENT{ Most years aren'{}t leap years... }\\
  isLeapYear\gets\FALSE\\
  \IF (year\bmod\ 4=0)\ \AND(year\bmod\ 100\neq\ 0) \THEN
  \BEGIN
    \COMMENT{ This is a standard leap year }\\
    isLeapYear\gets\TRUE\\
  \END\\
  \ELSE
    \IF year\bmod\ 400=0 \THEN
    \BEGIN
      \COMMENT{ One of the rare leap years }\\
      \COMMENT{ occurring every 400 years }\\
      isLeapYear\gets\TRUE\\
    \END\\
\ENDPROCEDURE
\end{pseudocode}


\begin{pseudocode}{daysInMonth423}{aDate }
\label{daysInMonth423}
\COMMENT{ Computes the number of days the given month (1..12) }\\
\COMMENT{ has in the the given year }\\
\PROCEDURE{daysInMonth423}{aDate}
  \COMMENT{ select the case where illegal values are also considered }\\
  discr7a3d45bd <- aDate.month\\
  \IF discr7a3d45bd=1\ \OR\ discr7a3d45bd=3\ \OR\ discr7a3d45bd=5\ \OR\ discr7a3d45bd=7\ \OR\ discr7a3d45bd=8\ \OR\ discr7a3d45bd=10\ \OR\ discr7a3d45bd=12 \THEN
    days\gets\ 31\\
  \ELSEIF discr7a3d45bd=4\ \OR\ discr7a3d45bd=6\ \OR\ discr7a3d45bd=9\ \OR\ discr7a3d45bd=11 \THEN
    days\gets\ 30\\
  \ELSEIF discr7a3d45bd=2 \THEN
  \BEGIN
    \COMMENT{ Default value for February }\\
    days\gets\ 28\\
    \COMMENT{ To make the call work it has to be done in }\\
    \COMMENT{ a separate element (cannot be performed }\\
    \COMMENT{ as part of the condition of an Alternative) }\\
    isLeap\gets\CALL{isLeapYear}{aDate.year}\\
    \IF isLeap \THEN
      days\gets\ 29\\
  \END\\
  \ELSE
  \BEGIN
    \COMMENT{ This is the return value for illegal months. }\\
    \COMMENT{ It is easy to check }\\
    days\gets\ 0\\
  \END\\
  \RETURN{days}\\
\ENDPROCEDURE
\end{pseudocode}


\COMMENT{ = = = = 8< = = = = = = = = = = = = = = = = = = = = = = = = = = = = = = }\\


\begin{pseudocode}{DateTests563}{ }
\label{DateTests563}
\MAIN
  Date\ someDay\gets\ Date\{day:24,month:2,year:2017\}\\
  nDays\gets\CALL{daysInMonth423}{someDay}\\
  today\gets\ Date\{2018,7,20\}\\
  type\ Person=record\{name:string;birth:Date;test:array[3]of\ int;\}\\
  var\ me:Person\gets\ Person\{\)"{}roger"{}\(,Date\{1985,3,6\},\{0,8,15\}\}\\
  var\ declArray:array\ of\ double\gets\{9.0,7.5,-6.4,1.7,0.0\}\\
  var\ explArray:double[3]\gets\{7.1,0.5,-1.5\}\\
  double\ doof[3]\gets\{0.4\}\\
  values\gets\{47,11\}\\
\ENDMAIN
\end{pseudocode}


\begin{pseudocode}{CommonTypes423}{ }
\label{CommonTypes423}
\COMMENT{ Provides type definitions for other programs }\\

  type\ Date=record\{year:int;month,day:short\}\\
  var\ today:Date\\

\end{pseudocode}

\end{document}
