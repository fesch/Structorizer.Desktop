\documentclass[a4paper,10pt]{article}

\usepackage[inoutnumbered]{algorithm2e}
\usepackage{ngerman}
\usepackage{amsmath}

\DeclareMathOperator{\oprdiv}{div}
\DeclareMathOperator{\oprshl}{shl}
\DeclareMathOperator{\oprshr}{shr}
\SetKwInput{Input}{input}
\SetKwInput{Output}{output}
\SetKwBlock{Parallel}{parallel}{end}
\SetKwFor{Thread}{thread}{:}{end}
\SetKwBlock{Try}{try}{end}
\SetKwFor{Catch}{catch (}{)}{end}
\SetKwBlock{Finally}{finally}{end}
\SetKwProg{Prog}{Program}{ }{end}
\SetKwProg{Proc}{Procedure}{ }{end}
\SetKwProg{Func}{Function}{ }{end}
\SetKwProg{Incl}{Includable}{ }{end}

\SetKw{preLeave}{leave}
\SetKw{preExit}{exit}
\SetKw{preThrow}{throw}
\DontPrintSemicolon
\title{Structorizer LaTeX pseudocode Export of SORTING\_TEST.arrz}
% Structorizer version 3.32-13
\author{Kay G"urtzig}
\date{17.10.2023}

\begin{document}
\LinesNumbered

\begin{procedure}
\caption{bubbleSort(values)}
\tcc{ Implements the well-known BubbleSort algorithm. }
\tcc{ Compares neigbouring elements and swaps them in case of an inversion. }
\tcc{ Repeats this while inversions have been found. After every }
\tcc{ loop passage at least one element (the largest one out of the }
\tcc{ processed subrange) finds its final place at the end of the }
\tcc{ subrange. }
\Proc{\FuncSty{bubbleSort(}\ArgSty{values}\FuncSty{)}}{
\KwData{\(values\): ?}
  \(ende\leftarrow{}length(values)-2\)\;
  \Repeat{\(posSwapped<0\)}{
    \tcc{ The index of the most recent swapping (-1 means no swapping done). }
    \(posSwapped\leftarrow{}-1\)\;
    \For{\(i\leftarrow 0\) \KwTo \(ende\) \textbf{by} \(1\)}{
      \If{\(values[i]>values[i+1]\)}{
        \(temp\leftarrow{}values[i]\)\;
        \(values[i]\leftarrow{}values[i+1]\)\;
        \(values[i+1]\leftarrow{}temp\)\;
        \(posSwapped\leftarrow{}i\)\;
      }
    }
    \(ende\leftarrow{}posSwapped-1\)\;
  }
}
\end{procedure}


\begin{procedure}
\caption{maxHeapify(heap, i, range)}
\SetKwFunction{FnmaxHeapify}{maxHeapify}
\tcc{ Given a max-heap '{}heap\textasciiacute{} with element at index '{}i\textasciiacute{} possibly }
\tcc{ violating the heap property wrt. its subtree upto and including }
\tcc{ index range-1, restores heap property in the subtree at index i }
\tcc{ again. }
\Proc{\FuncSty{maxHeapify(}\ArgSty{heap, i, range}\FuncSty{)}}{
\KwData{\(heap\): ?}
\KwData{\(i\): ?}
\KwData{\(range\): ?}
  \tcc{ Indices of left and right child of node i }
  \(right\leftarrow{}(i+1)*2\)\;
  \(left\leftarrow{}right-1\)\;
  \tcc{ Index of the (local) maximum }
  \(max\leftarrow{}i\)\;
  \If{\(left<range\wedge\ heap[left]>heap[i]\)}{
    \(max\leftarrow{}left\)\;
  }
  \If{\(right<range\wedge\ heap[right]>heap[max]\)}{
    \(max\leftarrow{}right\)\;
  }
  \If{\(max\neq\ i\)}{
    \(temp\leftarrow{}heap[i]\)\;
    \(heap[i]\leftarrow{}heap[max]\)\;
    \(heap[max]\leftarrow{}temp\)\;
    \(\FnmaxHeapify(heap,max,range)\)\;
  }
}
\end{procedure}


\begin{function}
\caption{partition(values, start, stop, p)}
\tcc{ Partitions array '{}values\textasciiacute{} between indices '{}start\textasciiacute{} und '{}stop\textasciiacute{}-1 with }
\tcc{ respect to the pivot element initially at index '{}p\textasciiacute{} into smaller }
\tcc{ and greater elements. }
\tcc{ Returns the new (and final) index of the pivot element (which }
\tcc{ separates the sequence of smaller elements from the sequence }
\tcc{ of greater elements). }
\tcc{ This is not the most efficient algorithm (about half the swapping }
\tcc{ might still be avoided) but it is pretty clear. }
\Func{\FuncSty{partition(}\ArgSty{values, start, stop, p}\FuncSty{)}:int}{
\KwData{\(values\): ?}
\KwData{\(start\): ?}
\KwData{\(stop\): ?}
\KwData{\(p\): ?}
\KwResult{int}
  \tcc{ Cache the pivot element }
  \(pivot\leftarrow{}values[p]\)\;
  \tcc{ Exchange the pivot element with the start element }
  \(values[p]\leftarrow{}values[start]\)\;
  \(values[start]\leftarrow{}pivot\)\;
  \(p\leftarrow{}start\)\;
  \tcc{ Beginning and end of the remaining undiscovered range }
  \(start\leftarrow{}start+1\)\;
  \(stop\leftarrow{}stop-1\)\;
  \tcc{ Still unseen elements? }
  \tcc{ Loop invariants: }
  \tcc{ 1. p = start - 1 }
  \tcc{ 2. pivot = values[p] }
  \tcc{ 3. i \textless start \textrightarrow{} values[i] ≤ pivot }
  \tcc{ 4. stop \textless i \textrightarrow{} pivot \textless values[i] }
  \While{\(start\leq\ stop\)}{
    \tcc{ Fetch the first element of the undiscovered area }
    \(seen\leftarrow{}values[start]\)\;
    \tcc{ Does the checked element belong to the smaller area? }
    \eIf{\(seen\leq\ pivot\)}{
      \tcc{ Insert the seen element between smaller area and pivot element }
      \(values[p]\leftarrow{}seen\)\;
      \(values[start]\leftarrow{}pivot\)\;
      \tcc{ Shift the border between lower and undicovered area, }
      \tcc{ update pivot position. }
      \(p\leftarrow{}p+1\)\;
      \(start\leftarrow{}start+1\)\;
    }{
      \tcc{ Insert the checked element between undiscovered and larger area }
      \(values[start]\leftarrow{}values[stop]\)\;
      \(values[stop]\leftarrow{}seen\)\;
      \tcc{ Shift the border between undiscovered and larger area }
      \(stop\leftarrow{}stop-1\)\;
    }
  }
  \Return{\(p\)}
}
\end{function}


\begin{function}
\caption{testSorted(numbers)}
\tcc{ Checks whether or not the passed-in array is (ascendingly) sorted. }
\Func{\FuncSty{testSorted(}\ArgSty{numbers}\FuncSty{)}:bool}{
\KwData{\(numbers\): ?}
\KwResult{bool}
  \(isSorted\leftarrow{}true\)\;
  \(i\leftarrow{}0\)\;
  \tcc{ As we compare with the following element, we must stop at the penultimate index }
  \While{\(isSorted\wedge(i\leq\ length(numbers)-2)\)}{
    \tcc{ Is there an inversion? }
    \eIf{\(numbers[i]>numbers[i+1]\)}{
      \(isSorted\leftarrow{}false\)\;
    }{
      \(i\leftarrow{}i+1\)\;
    }
  }
  \Return{\(isSorted\)}
}
\end{function}


\begin{procedure}
\caption{buildMaxHeap(heap)}
\SetKwFunction{FnmaxHeapify}{maxHeapify}
\tcc{ Runs through the array heap and converts it to a max-heap }
\tcc{ in a bottom-up manner, i.e. starts above the "{}leaf"{} level }
\tcc{ (index \textgreater= length(heap) div 2) and goes then up towards }
\tcc{ the root. }
\Proc{\FuncSty{buildMaxHeap(}\ArgSty{heap}\FuncSty{)}}{
\KwData{\(heap\): ?}
  \(lgth\leftarrow{}length(heap)\)\;
  \For{\(k\leftarrow lgth\oprdiv\ 2-1\) \KwTo \(0\) \textbf{by} \(-1\)}{
    \(\FnmaxHeapify(heap,k,lgth)\)\;
  }
}
\end{procedure}


\begin{procedure}
\caption{quickSort(values, start, stop)}
\SetKwFunction{Fnpartition}{partition}
\SetKwFunction{FnquickSort}{quickSort}
\tcc{ Recursively sorts a subrange of the given array '{}values\textasciiacute{}.  }
\tcc{ start is the first index of the subsequence to be sorted, }
\tcc{ stop is the index BEHIND the subsequence to be sorted. }
\Proc{\FuncSty{quickSort(}\ArgSty{values, start, stop}\FuncSty{)}}{
\KwData{\(values\): ?}
\KwData{\(start\): ?}
\KwData{\(stop\): ?}
  \tcc{ At least 2 elements? (Less don'{}t make sense.) }
  \If{\(stop\geq\ start+2\)}{
    \tcc{ Select a pivot element, be p its index. }
    \tcc{ (here: randomly chosen element out of start ... stop-1) }
    \(p\leftarrow{}random(stop-start)+start\)\;
    \tcc{ Partition the array into smaller and greater elements }
    \tcc{ Get the resulting (and final) position of the pivot element }
    \(p\leftarrow{}\Fnpartition(values,start,stop,p)\)\;
    \tcc{ Sort subsequances separately and independently ... }
    \Parallel{
      \Thread{1}{
        \tcc{ Sort left (lower) array part }
        \(\FnquickSort(values,start,p)\)\;
      }
      \Thread{2}{
        \tcc{ Sort right (higher) array part }
        \(\FnquickSort(values,p+1,stop)\)\;
      }
    }
  }
}
\end{procedure}


\begin{procedure}
\caption{heapSort(values)}
\SetKwFunction{FnmaxHeapify}{maxHeapify}
\SetKwFunction{FnbuildMaxHeap}{buildMaxHeap}
\tcc{ Sorts the array '{}values\textasciiacute{} of numbers according to he heap sort }
\tcc{ algorithm }
\Proc{\FuncSty{heapSort(}\ArgSty{values}\FuncSty{)}}{
\KwData{\(values\): ?}
  \(\FnbuildMaxHeap(values)\)\;
  \(heapRange\leftarrow{}length(values)\)\;
  \For{\(k\leftarrow heapRange-1\) \KwTo \(1\) \textbf{by} \(-1\)}{
    \(heapRange\leftarrow{}heapRange-1\)\;
    \tcc{ Swap the maximum value (root of the heap) to the heap end }
    \(maximum\leftarrow{}values[0]\)\;
    \(values[0]\leftarrow{}values[heapRange]\)\;
    \(values[heapRange]\leftarrow{}maximum\)\;
    \(\FnmaxHeapify(values,0,heapRange)\)\;
  }
}
\end{procedure}


\tcc{ = = = = 8< = = = = = = = = = = = = = = = = = = = = = = = = = = = = = = }


\begin{algorithm}
\caption{SORTING\_TEST\_MAIN}
\SetKwFunction{FnbubbleSort}{bubbleSort}
\SetKwFunction{FntestSorted}{testSorted}
\SetKwFunction{FnquickSort}{quickSort}
\SetKwFunction{FnheapSort}{heapSort}
\tcc{ Creates three equal arrays of numbers and has them sorted with different sorting algorithms }
\tcc{ to allow performance comparison via execution counting ("{}Collect Runtime Data"{} should }
\tcc{ sensibly be switched on). }
\tcc{ Requested input data are: Number of elements (size) and filing mode. }
\Prog{\FuncSty{SORTING\_TEST\_MAIN}}{
  \Repeat{\(elementCount\geq\ 1\)}{
    \Input{\(elementCount\)}
  }
  \Repeat{\(modus=1\vee\ modus=2\vee\ modus=3\)}{
    \Input{\(\)"{}Filling:\ 1\ =\ random,\ 2\ =\ increasing,\ 3\ =\ decreasing"{}\(,modus\)}
  }
  \For{\(i\leftarrow 0\) \KwTo \(elementCount-1\) \textbf{by} \(1\)}{
    \Switch{modus}{
      \Case{1}{
        \(values1[i]\leftarrow{}random(10000)\)\;
      }
      \Case{2}{
        \(values1[i]\leftarrow{}i\)\;
      }
      \Case{3}{
        \(values1[i]\leftarrow{}-i\)\;
      }
    }
  }
  \tcc{ Copy the array for exact comparability }
  \For{\(i\leftarrow 0\) \KwTo \(elementCount-1\) \textbf{by} \(1\)}{
    \(values2[i]\leftarrow{}values1[i]\)\;
    \(values3[i]\leftarrow{}values1[i]\)\;
  }
  \Parallel{
    \Thread{1}{
      \(\FnbubbleSort(values1)\)\;
    }
    \Thread{2}{
      \(\FnquickSort(values2,0,elementCount)\)\;
    }
    \Thread{3}{
      \(\FnheapSort(values3)\)\;
    }
  }
  \(ok1\leftarrow{}\FntestSorted(values1)\)\;
  \(ok2\leftarrow{}\FntestSorted(values2)\)\;
  \(ok3\leftarrow{}\FntestSorted(values3)\)\;
  \If{\(!ok1\vee!ok2\vee!ok3\)}{
    \For{\(i\leftarrow 0\) \KwTo \(elementCount-1\) \textbf{by} \(1\)}{
      \If{\(values1[i]\neq\ values2[i]\vee\ values1[i]\neq\ values3[i]\)}{
        \Output{\(\)"{}Difference\ at\ ["{}\(,i,\)"{}]:\ "{}\(,values1[i],\)"{}\ \textless-\textgreater\ "{}\(,values2[i],\)"{}\ \textless-\textgreater\ "{}\(,values3[i]\)}
      }
    }
  }
  \Repeat{\(show=\)"{}yes"{}\(\vee\ show=\)"{}no"{}\(\)}{
    \Input{\(\)"{}Show\ arrays\ (yes/no)?"{}\(,show\)}
  }
  \If{\(show=\)"{}yes"{}\(\)}{
    \For{\(i\leftarrow 0\) \KwTo \(elementCount-1\) \textbf{by} \(1\)}{
      \Output{\(\)"{}["{}\(,i,\)"{}]:\textbackslash{}t"{}\(,values1[i],\)"{}\textbackslash{}t"{}\(,values2[i],\)"{}\textbackslash{}t"{}\(,values3[i]\)}
    }
  }
}
\end{algorithm}

\end{document}
